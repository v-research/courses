\section{Conclusioni}
Abbiamo visto come la verifica formale dei protocolli di sicurezza sia un'operazione importante, da effettuare prima di utilizzare un protocollo all'interno di un'applicazione o di un software, per garantire la sicurezza dei dati e delle informazioni.\\
Inoltre abbiamo visto come allo stato dell'arte esistano il modello computazionale e il modello simbolico per la verifica formale automatica dei protocolli, ma solo il modello simbolico, il quale utilizza l'attaccante del modello Dolev-Yao, è abbastanza ``maturo'' per essere effettivamente utilizzato.\\ 
In questo documento è stato presentato un nuovo modo, intuitivo e agile, per modellare i protocolli mediante diagrammi UML, che può essere utilizzato dai progettisti ed è stata fatta un'analisi su limiti e capacità delle tecniche di modellazione attuali. \\
\`E stato presentato il software sviluppato appositamente per trasformare i protocolli modellati mediante diagrammi UML in un file utilizzabile come input per il tool di verifica automatica VerifPal.\\
I protocolli analizzati con VerifPal ci hanno consentito di analizzare limiti e capacità di questo tool di verifica automatica e di verificarne la bontà dei risultati di analisi.\\
Un obiettivo futuro può essere quello di utilizzare i software open source Modelio e VerifPal insieme al tool di conversione  presentato, per creare un unico software open source in grado di consentire ai progettisti di modellare i protocolli tramite diagrammi UML ed avere immediatamente garanzie sulla sicurezza dei protocolli.
