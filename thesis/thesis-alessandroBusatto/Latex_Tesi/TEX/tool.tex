\section{Tool per la verifica di protocolli basati sul modello simbolico}
\label{sez:tool}

Sviluppare un tool per la verifica automatica di protocolli basato sul modello simbolico resta comunque una sfida, questo perch\'e lo spazio di stati da esplorare è potenzialmente infinito per due motivi:
\begin{enumerate}
    \item la dimensione dello spazio dei messaggi non è definita in presenza di un attaccante attivo,
    \item il numero di sessioni del protocollo non è limitato.
\end{enumerate}
Una semplice soluzione a questo problema è quella di esplorare solo una parte dello spazio degli stati, limitando arbitrariamente sia la dimensione dello spazio dei messaggi che il numero di sessioni del protocollo.\\
Solo se il numero di sessioni è limitato, la verifica dei protocolli rimane decidibile: l'insicurezza del protocollo (esistenza di un attacco) è NP-completa con ipotesi ragionevoli sulle primitive crittografiche.\\
Nonostante questa indecidibilità, molte tecniche sono state progettate per verificare protocolli con un numero illimitato di sessioni, limitandosi a sottoclassi di protocolli, richiedendo l'interazione dell'utente, tollerando la non terminazione, o con sistemi incompleti (che possono rispondere ``Non lo so'').\\
In seguito verrano analizzati i tool per la verifica automatica dei protocolli basati sul modello simbolico ProVerif e VerifPal.\\

\subsection{ProVerif}\label{sub:pro}
Per la descrizione e l'analisi dei protocolli di sicurezza il tool ProVerif utilizza un linguaggio chiamato Applied Pi Calculus, un'estensione di Pi Calculus, il quale consente di dettagliare le azioni dei partecipanti al protocollo concentrando l'attenzione sulla loro comunicazione, più in generale consente di modellare processi concorrenti e la loro interazione.\\
A differenza del Pi Calculus vengono utilizzati termini al posto dei nomi per i messaggi.\\
Applied Pi Calculus \cite{AF16} è un linguaggio fortemente tipato che aggiunge a  Pi Calculus l'algebra utile a modellare le operazioni crittografiche utilizzate dai protocolli di sicurezza mediante una teoria equazionale (ad esempio l'operazione di modulo utilizzata nella generazione delle chiavi).\\
Inoltre consente di definire manualmente le primitive di sicurezza a differenza di un'altra estensione chiamata SPi Calculus \cite{AG97}, dove le primitive crittografiche come encryption e decryption sono implementate internamente.\\

\subsubsection*{Modellazione con ProVerif}

La modellazione di un protocollo in un file da dare in input a ProVerif può essere suddivisa in tre parti:
\begin{enumerate}
    \item dichiarazione formale del comportamento delle primitive crittografiche
    \item definizione di macroprocessi che consentono l'utilizzo di sotto-processi per semplificare lo sviluppo
    \item codifica del protocollo stesso come processo principale utilizzando le macro
\end{enumerate}

\subsubsection*{Dichiarazioni}
I processi sono composti da un insieme finito di tipi, nomi liberi e costruttori (funzioni simboliche) che
sono associati ad un insieme finito di decostruttori.\\ 
Il linguaggio è fortemente tipato e l'utente può definire dei nuovi tipi in questo modo: 
\begin{lstlisting}[language=app]
    type t . 
\end{lstlisting}
Tutti i nomi liberi devono essere dichiarati usando la sintassi: 
\begin{lstlisting}[language=app]
    free n : t .
\end{lstlisting} 
dove n è un nome e t il tipo.\\
La sintassi per dichiarare un canale è: 
\begin{lstlisting}[language=app]
    free c:channel.   
\end{lstlisting}
Tutte i nomi dichiarati con free sono conosciuti dall'attaccante, per fare in modo che non siano conosciuti dall'attaccante vanno dichiarati come privati utilizzando questa sintassi: 
\begin{lstlisting}[language=app]
    free n : t [ private ]. 
\end{lstlisting}
I costruttori (funzioni simboliche) sono usati per costruire termini di modellazione di primitive, usati dalla crittografia dei
protocolli, per esempio: funzioni di hash one-way, cifratura e firme digitali.\\
I costruttori si definiscono con:
\begin{lstlisting}[language=app, mathescape]
    fun $f(t_1,\dots, t_n )$ : t . 
\end{lstlisting} 
dove $f$ è un costruttore di arietà $n$, t è il tipo dell'oggetto di output e $t_1,\dots, t_n $ sono i tipi degli argomenti di input.\\
Anche i costruttori sono conosciuti dall'attaccante a meno della dichiarazione utilizzando [private].\\
I decostruttori vengono utilizzati per manipolare i termini formati dai costruttori e catturano le relazioni tra le primitive crittografiche, si modellano usando regole di riscrittura della forma:
\begin{lstlisting}[language=app, mathescape, breaklines= true]
    reduc forall $x_{1,1} : t_{1,1},\dots, x_{1,n_1}  : t_{1,n_1} ; 
    g(M_{1,1} , \dots , M_{1,k}) = M_{1,0} ;$
\end{lstlisting} 
dove $g$ è un decostruttore di arietà $k$, i termini $M_{1,1} , \dots , M_{1,k}, M_{1,0}$ sono costruiti dall'applicazione del costruttore alle variabili $x_{1,1},\dots, x_{1,n_1}$ di tipo $t_{1,1},\dots,t_{1,n_1}$ rispettivamente ed il tipo dell'output è $M_{1,0}$.\\
Analogamente ai costruttori, i decostruttori possono essere dichiarati privati con l'aggiunta di [private].\\
Vediamo come i costruttori e decostruttori vengono utilizzati per la definizione manuale delle primitive crittografiche:

\begin{lstlisting}[language=app]
    type key .
    fun senc(bitstring, key) : bitstring.
    reduc forall m: bitstring, k : key; 
        sdec(senc(m,k),k) = m.
\end{lstlisting}

\subsubsection*{Macro Processi}
Per semplificare lo sviluppo i sotto-processi vengono dichiarati utilizzando macro della forma: let $R(x_1:t_1,\dots,x_n:t_n) = P$., dove $R$ è il nome della macro, $P$ è il sotto-processo che si vuole definire e $x_1,\dots,x_n$ sono le variabili libere di $P$ di tipo $t_1,\dots,t_n$

\subsubsection*{Processi}

\begin{table}[h!]
    \begin{tabular}{ll}
        $M,N ::=$ & $termini$\\
        \quad$a, b, c, \dots , k, \dots , m, n, \dots , s$ & $nomi$\\
        \quad$x, y, z$ & $variabili$\\
        \quad$h(M_1, \dots , M_k)$ & $applicazione \: di \: costruttore/decostruttore$\\
        \quad $M=N$ & $uguaglianza \: tra \: termini$\\
        \quad $M<>N$ & $disuglianza \: tra \: termini$\\
        \quad $M\&\&N$ & $congiunzione$\\
        \quad $M||M$ & $disgiunzione$\\
        \quad not$M$ & $negazione$\\    
    \end{tabular}
    \caption{Grammatica di base}
    \label{tab:gb}
\end{table}

\begin{table}[h!]
    \begin{tabular}{ll}
        
        $P, Q::=$ & $processi$\\
        \quad$0$ & $processo \: vuoto$\\
        \quad$P|Q$ & $composizione \: parallela$\\
        \quad$!P$ & $replicazione$\\
        \quad new $n:t;P$ & $limitazione \: del \: nome$\\
        \quad$if \: M = N \: then \: P \: else \: Q$ & $condizione$\\
        \quad in$(M,x:t);P$ & $messaggio \: in \: input$\\
        \quad out$(M,N);P$ & $messaggio \: in \: output$\\
        \quad let $x=M \: in \: P \: else  \: Q$ & $valutazione \: del \: termine$\\
        \quad $R(M_1,\dots,M_k)$ & $utilizzo \: delle \: macro$\\       
    \end{tabular}
    \caption{Grammatica dei processi}
    \label{tab:gp}
\end{table}

\noindent Nella Tabella \ref{tab:gb} è descritta la grammatica di base dove i termini $M,N$ consistono in nomi $a, b, c, k, m, n, s$, variabili $x,y,z$, tuple ($M_1,\dots, M_l$) dove l è l'arietà della tupla, funzioni simboliche (costruttori/decostruttori) indicati con $h(M_1, \dots , M_k)$ dove k è l'arietà di h e gli argomenti $M_1,...,M_k$ sono del tipo richiesto.\\
Le altre funzioni utilizzano la notazione infissa e lavorano con l'algebra booleana.\\
Nella Tabella \ref{tab:gp} è descritta la grammatica dei processi dove il processo vuoto 0 non fa nulla, $P|Q$ è la composizione parallela di $P$ e $Q$, la replicazione $!P$ si comporta come un numero infinito di copie di $P$ in esecuzione in parallelo.\\ 
new $n:t;P$ lega il nome $n$ del tipo $t$ a $P$.\\
Il costrutto condizionale $if \: M = N \: then \: P \: else \: Q$ è standard e $M=N$ rappresenta l'uguaglianza, non l'identità.\\
Il processo in$(M,x:t);P$ indica che il processo $P$ è in attesa di un messaggio $x$ di tipo $t$ dal canale $M$ e poi prosegue la sua esecuzione.\\
Il processo out$(M,N);P$ indica che il processo $P$ è pronto per inviare $N$ nel canale $M$ e proseguire con la sua esecuzione.\\
Per evitare ambiguità durante l'esecuzione dei processi è consigliato utilizzare le parentesi per ordinarli.\\

\subsubsection*{Proprietà di sicurezza}
Per verificare la segretezza di un termine $M$ in un modello è sufficiente inserire la seguente query prima del processo principale:
\begin{lstlisting}[language=app]
    query attacker(M) .
\end{lstlisting} 
dove M è un nome di solito dichiarato privato, se così non fosse l'attaccante ne sarebbe banalmente a conoscenza.\\
Per verificare la proprietà di autenticazione è necessario annotare i processi con degli eventi in alcuni stati importanti, che non influiscono sul comportamento del protocollo.\\
Gli eventi verranno utilizzati per chiedere al tool se un determinato evento ha avuto luogo prima di un altro.\\
La sintassi per una richiesta di autenticazione è la seguente:
\begin{lstlisting}[language=app, mathescape]
    query $x_1:t_1 , \dots , x_n:t_n;$ 
    event $(e_1(M_1 ,\dots, M_j)) == >$ event $(e_0(N_1,\dots,N_k))$ .
\end{lstlisting} 
Ipotizzando che l'evento $e_0$ sia l'accettazione di un determinato client da parte di un server e l'evento $e_1$ sia l'invio di un messaggio dal client al server, con questo tipo di query ci chiediamo se il messaggio destinato al server è effettivamente stato inviato dal client corretto, nel caso in cui l'evento $e_0$ si sia effettivamente verificato prima dell'evento $e_1$, possiamo dire con certezza che il client è autenticato.\\
\noindent Ecco un esempio di come viene modellato in Applied Pi Calculus l'invio di un messaggio dall'agente A all'agente B e l'attaccante che prova ad intercettarlo:
\lstinputlisting[language=app, label={lst:hw}]{../code/hw.pv}


\subsection{VerifPal}
Tra i vari tool basati sul modello simbolico in grado di effettuare l'analisi formale dei protocolli di sicurezza, attualmente, il più utilizzato è ProVerif, purtroppo questo tool richiede all'utilizzatore di conoscere la sintassi del linguaggio Applied Pi Calculus, la quale è poco intuitiva, di conoscere il funzionamento delle clausole Horn create implicitamente dal tool e delle clausole Horn che devono essere esplicitate per modellare in maniera corretta lo scenario in cui verificare il protocollo.\\
Per questi motivi è stato sviluppato un nuovo tool chiamato VerifPal.\\
L'obiettivo di VerifPal è quello di descrivere protocolli con un linguaggio molto simile a come i protocolli potrebbero essere descritti in una conversazione informale.\\
Per fare ciò, pur basandosi internamente su costruzione e decostruzione di termini astratti simili a ProVerif, fa si che l'utente debba solo definire gli agenti partecipanti al protocollo che hanno stati indipendenti, conoscono determinati valori ed eseguono operazioni con le primitive crittografiche.\\
Le primitive crittografiche sono già definite e devono essere solo richiamate, in questo modo si evita che l'utente possa implementarle in maniera errata.\\
VerifPal \`e nato con lo scopo di creare un tool utilizzabile nell'ingegneria, per questo si presta ad essere utilizzato per il raggiungimento dell'obiettivo principale di questo documento, ovvero quello di utilizzare la modellazione UML per avvicinare la verifica di protocolli all'ingegneria di sistemi.  

\subsection*{Modellazione con VerifPal}
Quando si scrive il codice per la verifica formale di protocolli da analizzare con il tool VerifPal, la prima cosa da fare è scegliere se l'attaccante è di tipo attivo o di tipo passivo.\\
L'attaccante di tipo attivo viene comunemente chiamato attaccante di tipo Dolev-Yao (Sezione \ref{sec:dy}), mentre l'attaccante di tipo passivo è semplicemente un ascoltatore in grado di vedere i pacchetti che transitano sulla rete.\\
La seconda cosa da fare è inizializzare gli agenti che partecipano al protocollo, chiamati \texttt{principal}\footnote{questo tipo di font verrà applicato a tutte le keyword utilizzate da VerifPal}, all'interno della definizione dei principal possono essere definite delle costanti.\\ 
Queste costanti possono essere conoscenze pregresse del principal, per fare questo vengono inizializzate con la parola chiave \texttt{knows}, oppure possono essere delle costanti con dei valori generati al momento, in questo caso si utilizza la parola chiave \texttt{generates}.\\
Inoltre, le costanti possono essere definite pubbliche o private, nel caso in cui la costante sia definita pubblica anche l'attaccante ne è a conoscenza.\\
Per quanto detto sopra, per evitare errori da parte dell'utente, le variabili hanno uno scope globale, di conseguenza non possono esistere due costanti con lo stesso nome, a meno che non vengano definite come \texttt{private} e non possono essere riassegnate ad altri valori.\\
Una volta inizializzati i principal si passa alla modellazione del protocollo vero e proprio, con lo scambio dei messaggi e le varie operazioni fatte dai principal.\\
Per inviare un messaggio basta semplicemente indicare mittente, destinatario e contenuto del messaggio in questo modo: 
\begin{lstlisting}[mathescape]
    A$\rightarrow$B: m 
\end{lstlisting}
 Inoltre è possibile forzare il fatto che l'attaccante attivo non possa modificare il messaggio m semplicemente inserendolo tra $[$ $]$, l'utilizzo di questa tecnica di guardia è sconsigliato, in quanto potrebbe alterare il risultato della verifica, se non utilizzata correttamente.\\
Come detto sopra VerifPal ci viene incontro definendo le seguenti primitive (le quali corrispondono anche alle primitive del modello Dolev-Yao): 
\begin{itemize}
    \item \textbf{\texttt{CONCAT}}(a,b...) : c $\rightarrow$ concatena due o più valori in uno
    \item \textbf{\texttt{SPLIT}}(\textbf{\texttt{CONCAT}}(a,b...)) : a,b. $\rightarrow$ separa una concatenazione negli elementi che la compongono
    \item \textbf{\texttt{ENC}}(key,paintext) : ciphertext $\rightarrow$ codifica nella crittografia a chiave simmetrica
    \item \textbf{\texttt{DEC}}(key, \textbf{\texttt{ENC}}(key,paintext)): plaintext $\rightarrow$ decodifica nella crittografia a chiave simmetrica
    \item \textbf{\texttt{PKE\_ENC}}($G^{key}$, plaintext) : ciphertext $\rightarrow$ codifica nella crittografia a chiave asimmetrica
    \item \textbf{\texttt{PKE\_DEC}}(key, \textbf{\texttt{PKE\_ENC}}($G^{key}$, plaintext)) : plaintext $\rightarrow$ decodifica nella crittografia a chiave asimmetrica
    \item \textbf{\texttt{SIGN}}(key, message) : signature $\rightarrow$ firma un messaggio
    \item \textbf{\texttt{SIGNVERIF}}($G^{key}$, message, \textbf{\texttt{SIGN}}(key, message)) : message $\rightarrow$ verifica della firma
\end{itemize}

\noindent Oltre a queste primitive VerifPal fornisce altre primitive utili per effettuare l'hash dei messaggi.\\
Il modello del protocollo si conclude con un blocco chiamato \texttt{queries}, dove sono contenute le domande alle quali vorremmo che VerifPal ci desse le risposte come risultato dell'analisi del modello.\\
Nel blocco di queries possiamo fare delle domande su confidenzialità, autenticazione, freshness dei messaggi utilizzando rispettivamente i comandi \texttt{confidentiality?}, \texttt{authentication?} e \texttt{freshness?}.\\
Inoltre nel caso in cui si volesse modellare uno scenario in cui l'attaccante riesce ad ottenere una costante dichiarata come privata in qualche modo, basta utilizzare la parola chiave \texttt{leaks} seguita dal nome della variabile.\\
Qui è proposto lo stesso scenario modellato nella Sezione \ref{lst:hw} con ProVerif:
\lstinputlisting[language=vp]{../code/test.vp}


