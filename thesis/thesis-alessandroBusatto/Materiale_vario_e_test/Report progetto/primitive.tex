\subsection{Le primitive nella modellazione UML}

Le primitive descritte nella Figura \ref*{fig:rdy} vengono rappresentate anche nella modellazione del protocollo attraverso l'utilizzo dei diagrammi UML.\\
La primitiva $G_{concat}$ (concatenazione) appare nei diagrammi UML quando abbiamo più input in ingresso ad un oggetto con un singolo output, ad esempio viene utilizzata quando in un oggetto viene preparato un pacchetto, dati vari parametri in ingresso vengono concatenati prima di essere inoltrati.\\
Al contrario la primitiva $A_{concat}$ (deconcatenazione) appare quando abbiamo un singolo input e in uscita più output, come quando abbiamo in ingresso un pacchetto e dobbiamo ricavare qualche elemento da cui è composto.\\
Nella modellazione UML gli oggetti di encryption e decryption sono considerati delle black box, questo perchè in caso di crittografia simmetrica corrispondono alle primitive $G_{critS}$ e $A_{critS}$ e in caso di crittografia asimmetrica corrispondono alle primitive $G_{crittA}$ e $A_{crittA}$ del modello Dolev-Yao.\\
Nel caso in cui nel diagramma sia rappresentato un oggetto per la firma di un messaggio o per la verifica della firma, ancora una volta troviamo nel modello Dolev-Yao le primitive necessarie, ovvero la primitiva  $G_{crittA}$, la quale si occupa della firma di un messaggio prendendo come input la chiave privata e il messaggio, e la primitiva $A^{-1}_{crittA}$, la quale verifica la firma utilizzando la chiave pubblica.


%Una volta disegnato il modello del protocollo attraverso la rappresentazione UML, è importante capire come le varie operazioni effettuate nel protocollo vengono tradotte in primitive del modello Dolev-Yao.\\
%Le primitive del modello Dolev-yao sono rappresentate nella Figura \ref*{fig:rdy}.
%La composizione e decomposizione di messaggi corrispondono alle funzioni $G_{concat}$ e $A_{concat}$, per quanto riguarda i blocchi in cui si fa l'encryption e la decryption dei messaggi essendo delle black box nel modello simbolico ed essendo presenti tra le primitive di Dolev-Yao utilizzo le funzioni $G_{critS}$ e $A_{critS}$ per la crittografia simmetrica, mentre utilizzo $G_{crittA}$ e $A_{crittA}$ per la crittografia asimmetrica.
%Nel caso in cui ci sia un blocco in cui voglio firmare un messaggio utilizzo la funzione $G_{crittA}$ invertendo chiave privata e chiave pubblica e la regola $A^{-1}_{crittA}$ per verificare la firma.\\


% •  Primitive cosa sono\\
% •  "Primitive" di encryption e decryption, come potrebbero essere rappresentate attraverso altre primitive e perchè con la semantica utilizzata vengono definite "primitive" (enc/dec vs concat/deconcat)
% \\•  Spiegazione dello stato dell'arte e del motivo per cui per la verifica formale dei protocolli si è scelto di utilizzare il modello di attaccante Dolev Yao rispetto all'obiettivo di trovare falle nei protocolli a prescindere dal tipo di attaccante
