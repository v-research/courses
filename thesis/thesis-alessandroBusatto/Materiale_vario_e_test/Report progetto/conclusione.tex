\section{Conclusioni}
Abbiamo visto come la verifica formale dei protocolli di sicurezza sia un'operazione fondamentale, da effettuare prima di utilizzare un protocollo all'interno di applicazione o di un software, per garantire la sicurezza dei dati e delle informazioni.\\
Inoltre abbiamo visto come allo stato dell'arte esistano due modelli per la verifica formale automatica dei protocolli, ma solo uno è abbastanza maturo per essere effettivamente utilizzato.\\
In questo documento è stato presentato un nuovo modo, più intuitivo e più veloce di quelli esistenti, per modellare i protocolli che può essere utilizzato dai progettisti ed è stata fatta un'analisi su limiti e capacità delle tecniche di modellazione attuali. \\
Un obiettivo per il futuro è quello di creare un software in grado di utilizzare il file .xmi estratto dalla modellazione UML per generare il file da usare come input per i tool di verifica ProVerif e VerifPal.