\subsubsection*{Modellazione con ProVerif}

La modellazione di un protocollo in un file da dare in input a ProVerif può essere suddivisa in tre parti:
%Il linguaggio di input per scrivere un modello di un protocollo in ProVerif è Applied Pi Calculus (un estensione tipata di pi calculus).\\
%La modellazione può essere divisa in tre parti:
\begin{enumerate}
    \item dichiarazione formale del comportamento delle primitive crittografiche
    \item definizione di macroprocessi che consentono l'utilizzo di sotto-processi per semplificare lo sviluppo
    \item codifica del protocollo stesso come processo principale utilizzando le macro
\end{enumerate}

\subsubsection*{Dichiarazioni}
I processi sono composti da un insieme finito di tipi, nomi liberi e costruttori (funzioni simboliche) che
sono associati ad un insieme finito di decostruttori.\\ 
Il linguaggio è fortemente tipato e l'utente può definire dei nuovi tipi in questo modo: 
\begin{lstlisting}[language=app]
    type t . 
\end{lstlisting}
Tutti i nomi liberi devono essere dichiarati usando la sintassi: 
\begin{lstlisting}[language=app]
    free n : t .
\end{lstlisting} 
dove n è un nome e t il tipo.\\
La sintassi per dichiarare un canale è: 
\begin{lstlisting}[language=app]
    free c:channel.   
\end{lstlisting}
Tutte i nomi dichiarati con free sono conosciuti dall'attaccante, per fare in modo che non siano conosciuti dall'attaccante vanno dichiarati come privati utilizzando questa sintassi: 
\begin{lstlisting}[language=app]
    free n : t [ private ]. 
\end{lstlisting}
I costruttori (funzioni simboliche) sono usati per costruire termini di modellazione di primitive, usati dalla crittografia dei
protocolli, per esempio: funzioni di hash one-way, cifratura e firme digitali.\\
I costruttori si definiscono con:
\begin{lstlisting}[language=app, mathescape]
    fun $f(t_1,\dots, t_n )$ : t . 
\end{lstlisting} 
dove $f$ è un costruttore di arietà $n$, t è il tipo dell'oggetto di output e $t_1,\dots, t_n $ sono i tipi degli argomenti di input.\\
Anche i costruttori sono conosciuti dall'attaccante a meno della dichiarazione utilizzando [private].\\
I decostruttori vegono utilizzati per manipolare i termini formati dai costruttori e catturano le relazioni tra le primitive crittografiche, si modellano usando regole di riscrittura della forma:
\begin{lstlisting}[language=app, mathescape, breaklines= true]
    reduc forall $x_{1,1} : t_{1,1},\dots, x_{1,n_1}  : t_{1,n_1} ; 
    g(M_{1,1} , \dots , M_{1,k}) = M_{1,0} ;$
\end{lstlisting} 
dove $g$ è un decostruttore di arietà $k$, i termini $M_{1,1} , \dots , M_{1,k}, M_{1,0}$ sono costruiti dall'applicazione del costruttore alle variabili $x_{1,1},\dots, x:{1,n_1}$ di tipo $t_{1,1},\dots,t_{1,n_1}$ rispettivamente ed il tipo dell'output è $M_{1,0}$.\\
Analogamente ai costruttori, i decostruttori possono essere dichiarati privati con l'aggiunta di [private].\\
Vediamo come i costruttori e decostruttori vengono utilizzati per la definizione manuale delle primitive crittografiche:
%A differenza di un'altra estensione chiamata spi calculus \cite{AG97}, nella quale le primitive di sicurezza come encryption e decryption sono delle primitive già definite, Applied Pi Calculus consente di definire manualmente le primitive di sicurezza andando ad utilizzare costruttori e decostruttori in questo modo:
%enc(x, pk(k)) = y
%dec(enc(x, pk(k)), k) = x
\begin{lstlisting}[language=app]
    type key .
    fun senc(bitstring, key) : bitstring.
    reduc forall m: bitstring, k : key; 
        sdec(senc(m,k),k) = m.
\end{lstlisting}

\subsubsection*{Macro Processi}
Per semplificare lo sviluppo i sotto-processi vengono dichiarati utilizzando macro della forma: let $R(x_1:t_1,\dots,x_n:t_n) = P$., dove $R$ è il nome della macro, $P$ è il sotto-processo che si vuole definire e $x_1,\dots,x_n$ sono le variabili libere di $P$ di tipo $t_1,\dots,t_n$

\subsubsection*{Processi}

\begin{table}[h!]
    \begin{tabular}{ll}
        $M,N ::=$ & $termini$\\
        \quad$a, b, c, \dots , k, \dots , m, n, \dots , s$ & $nomi$\\
        \quad$x, y, z$ & $variabili$\\
        \quad$h(M_1, \dots , M_k)$ & $applicazione \: di \: costruttore/decostruttore$\\
        \quad $M=N$ & $uguaglianza tra termini$\\
        \quad $M<>N$ & $disuglianza tra termini$\\
        \quad $M\&\&N$ & $congiunzione$\\
        \quad $M||M$ & $disgiunzione$\\
        \quad not$M$ & $negazione$\\    
    \end{tabular}
    \caption{Grammatica di base}
    \label{tab:gb}
\end{table}

\begin{table}[h!]
    \begin{tabular}{ll}
        
        $P, Q::=$ & $processi$\\
        \quad$0$ & $processo \: vuoto$\\
        \quad$P|Q$ & $composizione \: parallela$\\
        \quad$!P$ & $replicazione$\\
        \quad new $n:t;P$ & $limitazione \: del \: nome$\\
        \quad$if \: M = N \: then \: P \: else \: Q$ & $condizione$\\
        \quad in$(M,x:t);P$ & $messaggio \: in \: input$\\
        \quad out$(M,N);P$ & $messaggio \: in \: output$\\
        \quad let $x=M \: in \: P \: else  \: Q$ & $valutazione \: del \: termine$\\
        \quad $R(M_1,\dots,M_k)$ & $utilizzo \: delle \: macro$\\       
    \end{tabular}
    \caption{Grammatica dei processi}
    \label{tab:gp}
\end{table}

\noindent Nella Tabella \ref*{tab:gb} è descritta la grammatica di base dove i termini $M,N$ consistono in nomi $a, b, c, k, m, n, s$, variabili $x,y,z$, tuple ($M_1,\dots, M_l$) dove l è l'arietà della tupla, funzioni simboliche (costruttori/decostruttori) indicati con $h(M_1, \dots , M_k)$ dove k è l'arietà di h e gli argomenti $M_1,...,M_k$ sono del tipo richisto.\\
Le altre funzioni utilizzano la notazione infissa e lavorano con l'algebra booleana.\\
Nella Tabella \ref*{tab:gp} è descritta la grammatica dei processi dove il processo vuoto 0 non fa nulla, $P|Q$ è la composizione parallela di $P$ e $Q$; la replicazione $!P$ si comporta come un numero infinito di copie di $P$ in esecuzione in parallelo.\\ 
new $n:t;P$ lega il nome $n$ del tipo $t$ a $P$.\\
Il costrutto condizionale $if \: M = N \: then \: P \: else \: Q$ è standard e $M=N$ rappresenta l'uguaglianza, non l'identità.\\
Il processo in$(M,x:t);P$ indica che il processo $P$ è in attesa di un messaggio $x$ di tipo $t$ dal canale $M$ e poi prosegue la sua esecuzione.\\
Il processo out$(M,N);P$ indica che il processo $P$ è pronto per inviare $N$ nel canale $M$ e proseguire con la sua esecuzione.\\
Per evitare amibiguità durante l'esecuzione dei processi è consigliato utilizzare le parentesi per ordinarli.\\

\subsubsection*{Proprietà di sicurezza}
Per verificare la segretezza di un termine $M$ in un modello è sufficiente inserire la seguente query prima del processo principale:
\begin{lstlisting}[language=app]
    query attacker(M) .
\end{lstlisting} 
dove M è un nome di solito dichiarato privato, se così non fosse l'attaccante ne sarebbe banalmente a conoscenza.\\
Per verificare la proprietà di autenticazione è necessario annotare i processi con degli eventi in alcuni stati importanti, che non influiscono sul comportamento del protocollo.\\
Gli eventi verranno utilizzati per chiedere al tool se un determinato evento ha avuto luogo prima di un altro.\\
La sintassi per una richiesta di autenticazione è la seguente:
\begin{lstlisting}[language=app, mathescape]
    query $x_1:t_1 , \dots , x_n:t_n;$ 
    event $(e_1(M_1 ,\dots, M_j)) == >$ event $(e_0(N_1,\dots,N_k))$ .
\end{lstlisting} 
Ipotizzando che l'evento $e_0$ sia l'accettazione di un determinato client da parte di un server e l'evento $e_1$ sia l'invio di un messaggio dal client al server, con questo tipo di query ci chiediamo se il messaggio destinato al server è effettivamente stato inviato dal client corretto, nel caso in cui l'evento $e_0$ si sia effettivamente verificato prima dell'evento $e_1$, possiamo dire con certezza che il client è autenticato.\\

